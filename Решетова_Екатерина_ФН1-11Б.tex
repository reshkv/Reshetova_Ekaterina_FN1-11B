%------------------Settings-------------------------

\documentclass[12pt]{article}
\usepackage[utf8]{inputenc}
\usepackage[russian]{babel}
\usepackage{amsmath,amssymb}
\usepackage{graphics}
\usepackage{pbox}
\usepackage[x11names]{xcolor}
\definecolor{brightmaroon}{rgb}{0.76, 0.13, 0.28}
\definecolor{royalazure}{rgb}{0.0, 0.22, 0.66}
\usepackage[colorlinks=true,linkcolor=royalazure]{hyperref}
\usepackage{tikz, tkz-fct, pgfplots}
\usetikzlibrary{arrows}
\usepackage{geometry}
\geometry{
	a4paper,
	total={170mm,257mm},
	left=20mm,
	top=20mm
} 
\usepackage[labelsep=period]{caption}

% ----------------- Commands ----------------- 

\newcommand{\eps}{\varepsilon}
\newcommand\tline[2]{$\underset{\text{#1}}{\text{\underline{\hspace{#2}}}}$}

% ----------------- Set graphics path ----------------- 
\graphicspath{{img/}}
\begin{document}
\pagestyle{empty}

% ----------------------Title----------------------------------
\centerline{\large Министерство науки и высшего образования}	
\centerline{\large Федеральное государственное бюджетное образовательное}
\centerline{\large учреждение высшего образования}
\centerline{\large ``Московский государственный технический университет}
\centerline{\large имени Н.Э. Баумана}
\centerline{\large (национальный исследовательский университет)''}
\centerline{\large (МГТУ им. Н.Э. Баумана)}
\hrule
\vspace{0.5cm}
\begin{figure}[h]
\center
\includegraphics[height=0.35\linewidth]{bmstu-logo-small.png}
\end{figure}
\begin{center}
	\large	
	\begin{tabular}{c}
		Факультет ``Фундаментальные науки'' \\
		Кафедра ``Высшая математика''		
	\end{tabular}
\end{center}
\vspace{0.5cm}
\begin{center}
	\LARGE \bf	
	\begin{tabular}{c}
		\textsc{Отчёт} \\
		по учебной практике \\
		за 1 семестр 2020---2021 гг.
	\end{tabular}
\end{center}
\vspace{0.5cm}
\begin{center}
	\large
	\begin{tabular}{p{5.3cm}ll}
		\pbox{5.45cm}{
			Руководитель практики,\\
			ст. преп. кафедры ФН1} 	& \tline{\it(подпись)}{5cm} & Кравченко О.В. \\[0.5cm]
		студент группы ФН1--11 		& \tline{\it(подпись)}{5cm} & Решетова Е.В.
	\end{tabular}
\end{center}
\vfill
\begin{center}
	\large	
	\begin{tabular}{c}
		Москва, \\
		2020 г.
	\end{tabular}
\end{center}
 \newpage
 \newpage	
\tableofcontents

%------------------Table of contents----------------------
\newpage
\section{Цели и задачи практики}	
\subsection{Цели}
--- развитие компетенций, способствующих успешному освоению материала бакалавриата и необходимых в будущей профессиональной деятельности.

\subsection{Задачи}
\begin{enumerate}
\item Знакомство с программными средствами, необходимыми в будущей профессиональной деятельности.
\item Развитие умения поиска необходимой информации в специальной литературе и других источниках.
\item Развитие навыков составления отчётов и презентации результатов.
\end{enumerate}

\subsection{Индивидуальное задание}	
\begin{enumerate}
\item Изучить способы отображения математической информации в системе вёртски \LaTeX.
\item Изучить возможности  системы контроля версий \textsf{Git}.
\item Научиться верстать математические тексты, содержащие формулы и графики в системе \LaTeX.
Для этого, выполнить установку свободно распространяемого дистрибутива \textsf{TeXLive} и оболочки \textsf{TeXStudio}.
\item Оформить в системе \LaTeX типовые расчёты по курсе математического анализа согласно своему варианту.
\item Создать аккаунт на онлайн ресурсе \textsf{GitHub} и загрузить исходные \textsf{tex}--файлы 
и результат компиляции в формате \textsf{pdf}.
\end{enumerate}
 
%---------------------------------------------------------------
\newpage
\section{Отчёт}
Актуальность темы продиктована необходимостью владеть системой вёрстки \LaTeX и средой вёрстки \textsf{TeXStudio} для
отображения текста, формул и графиков. Полученные в ходе практики навыки могут быть применены при написании
курсовых проектов и дипломной работы, а также в дальнейшей профессиональной деятельности.

Ситема вёрстки \LaTeX содержит большое количество инструментов (пакетов), упрощающих отображение информации в различных 
сферах инженерной и научной деятельности. 
%-----------------------------------------------------------------
\newpage
\section{Индивидуальное задание}
%\subsection{Элементарные функции и их графики.}
%\input{src/part1.tex}

%---------------------------------------------------------------
\subsection{Пределы и непрерывность.}

%---------------------------- Problem 1----------------------------------
\subsubsection*{\center Задача № 1.}
{\bf Условие.~}
Дана последовательность $a_{n}=\dfrac{3-n^2}{1+2n^2}$ и число $c=-\dfrac{1}{2}$. Доказать, что $\lim\limits_{x\rightarrow\infty} a_{n}=c $, а именно, для каждого $\varepsilon>0$ найти наименьшее натуральное число  $N{=}N(\varepsilon)$ такое, что $|a_{n}-c|<\varepsilon$ для всех $n>N(\varepsilon)$. Заполнить таблицу: 
\begin{center}
\begin{tabular}{ | p{25pt} | c | c | c | c |}
\hline
$\varepsilon$& $0{,}1$ & $0{,}01$ & $0{,}001$ \\ \hline
$N(\varepsilon)$ &   &   &\\
\hline
\end{tabular}
\end{center}
\medskip
%=====================================================================
{\bf Решение.~}
Рассмотрим неравенство $a_{n}-c<\varepsilon$, $\forall\varepsilon>0$, учитывая выражение для $a_{n}$ и $c$ из условия варианта, получим 
$$\left|\frac{3-n^2}{1+2n^2}+\frac{1}{2}\right|<\varepsilon$$
Неравенство запишем в виде двойного неравенства и приведём выражение под знаком модуля к общему знаменателю, получим
$${-}\varepsilon <\dfrac{7}{2+4n^2}<\varepsilon$$
Заметим, что левое неравенство выполнено для любого номера $n\in \mathbb{N}$ поэтому, будем рассматривать правое неравенство
$$\frac{7}{2+4n^2}<\varepsilon$$
Выполнив цепочку преобразований, перепишем неравенство относительно $n$, и, учитывая, что $n\in \mathbb{N}$, получим 
$$\dfrac{7}{2(1+2n^2)}<\varepsilon,$$
$$1+2n^2>\dfrac{7}{2\varepsilon},$$
$$n>\sqrt{\dfrac{7}{4\varepsilon}-\dfrac{1}{2}},$$
$$n>\sqrt{\dfrac{7-2\varepsilon}{4\varepsilon}},$$
$$N(\varepsilon)=\biggl[\sqrt{\dfrac{7-2\varepsilon}{4\varepsilon}}\biggr],$$
где $[\;]$ -- целая часть от числа. Заполним таблицу:
\begin{center}
\begin{tabular}{ | p{25pt} | c | c | c | c |}
\hline
$\varepsilon$& $0{,}1$ & $0{,}01$ & $0{,}001$ \\ \hline
$N(\varepsilon)$ & 4  & 13 & 41\\
\hline
\end{tabular}
\end{center}
{\bf Проверка:~}
$$|a_{5}-c|=\dfrac{7}{102}<0{,}1,$$
$$|a_{14}-c|=\dfrac{7}{786}<0{,}01,$$
$$|a_{42}-c|=\dfrac{7}{7058}<0{,}001.$$
\newpage
% ---------------------------- Problem 2----------------------------------
\subsubsection*{\center Задача № 2.}
{\bf Условие.~}
Вычислить пределы функций
$$
\begin{array}{cc}
\text{\bf(а):} &  \lim\limits_{x\rightarrow-2}\bigg(\dfrac{-2}{x+2}+\dfrac{x^3}{x^2-4}\bigg) , \\[10pt]
\text{\bf(б):} & \lim\limits_{x\rightarrow+\infty} \dfrac{3x^4+2x^2\sqrt{x^6-x^3+1}}{\Big(\sqrt[3]{x^4+4}-\sqrt{x^5}\Big)^2} ,\\[10pt]
\text{\bf(в):} & \lim\limits_{x\rightarrow0} \dfrac{\sqrt{x+2}-\sqrt{2-x}}{\sqrt[3]{x+2}-\sqrt[3]{2-x}} ,\\[10pt]
\text{\bf(г):} & \lim\limits_{x\rightarrow1} {\Big(1+\tg{\pi{x}}\Big)}^{\dfrac{1}{x-1}} , \\[10pt]
\text{\bf(д):} & \lim\limits_{x\rightarrow0} {\big(\cos{x}\big)}^{\dfrac{3x^2}{1-\cos{x}}} , \\[10pt]
\text{\bf(е):}  & \lim\limits_{x\rightarrow\pi} \dfrac{x^2-\pi^2}{\sin{x}} . \\
\end{array}
$$
\\
{\bf Решение.~}\\
\\
\text{\bf(а):}
$$
\begin{array}{l}
\lim\limits_{x\rightarrow-2} \bigg(\dfrac{-2}{x+2}+\dfrac{x^3}{x^2-4}\bigg) =  \lim\limits_{x\rightarrow-2}  \bigg(\dfrac{(x+2)(x^2-2x+2)}{(x-2)(x+2)}\bigg) = \lim\limits_{x\rightarrow-2}  \bigg(\dfrac{x^2-2x+2}{x-2}\bigg) =-\dfrac{5}{2}
\end{array}
$$
\\
\text{\bf(б):}
$$
\begin{array}{l}
\lim\limits_{x\rightarrow+\infty} \dfrac{3x^4+2x^2\sqrt{x^6-x^3+1}}{\Big(\sqrt[3]{x^4+4}-\sqrt{x^5}\Big)^2}=\left[\dfrac{\infty}{\infty}\right]= \lim\limits_{x\rightarrow+\infty} \dfrac{3x^4+2\sqrt{x^{10}-x^7+x^4}}{\sqrt[3]{x^8+8x^4+16}-2\sqrt[3]{x^4+4}\sqrt{x^5}+x^5}=\\
 =\lim\limits_{x\rightarrow+\infty} \dfrac{\dfrac{3}{x}+2\sqrt{1-\dfrac{1}{x^3}+\dfrac{1}{x^6}}}{\sqrt[3]{\dfrac{1}{x^7}+\dfrac{8}{x^{11}}+\dfrac{16}{x^{15}}}-2\sqrt[3]{\dfrac{1}{x^{11}}+\dfrac{4}{x^{15}}}\sqrt{\dfrac{1}{x^5}}+1}
= \lim\limits_{x\rightarrow+\infty} \dfrac{0+2\sqrt{1-0+0}}{\sqrt[3]{0+0+0}-2\sqrt[3]{0+0}\sqrt{0}+1}=2
\end{array}
$$
\text{\bf(в):}
 $$
 \begin{array}{l}  
 \lim\limits_{x\rightarrow0} \dfrac{\sqrt{x+2}-\sqrt{2-x}}{\sqrt[3]{x+2}-\sqrt[3]{2-x}} = \left[\dfrac{0}{0} \right]= \lim\limits_{x\rightarrow0} \dfrac{2x\Big(\big(\sqrt[3]{x+2}\big)^2+\sqrt[3]{4-x^2}+\big(\sqrt[3]{2-x}\big)^2\Big)}{2x\Big(\sqrt{x+2}+\sqrt{2-x}\Big)} = \lim\limits_{x\rightarrow0} \dfrac{\big(2\sqrt[3]{2}\big)^2+\sqrt[3]{4}}{2\sqrt{2}}=\\ \medskip{}{} =\dfrac{3\sqrt[6]{2}}{2}
 \end{array}
 $$
 \\
\text{\bf(г):}
$$
\begin{array}{l}
 \lim\limits_{x\rightarrow1} \Big(1+\tg{\pi{x}}\Big)^{\dfrac{1}{x-1}}=\left[1^{\infty}\right] =\left|y=x-1,y\rightarrow0\right|=\lim\limits_{y\rightarrow0} \Big(1+\tg\left({\pi{y}+\pi}\right)\Big)^{\dfrac{1}{y}} = e^{\lim\limits_{y\rightarrow0} \dfrac{1}{y}\big(\tg{\pi{y}}\big)} = \\
=\left|y\rightarrow0,\tg{\pi{y}}\sim{\pi{y}}\right|=e^{\pi}
\end{array}
$$
\\
\text{\bf(д):}
$$
\begin{array}{l}
 \lim\limits_{x\rightarrow0} \left(\cos{x}\right)^{\dfrac{3x^2}{1-\cos{x}}}=\left[1^{\infty}\right]=e^{\lim\limits_{x\rightarrow0} \Big(\dfrac{3x^2}{1-\cos{x}}\Big)(\cos{x}-1)}=e^{-\lim\limits_{x\rightarrow0} \Big(\dfrac{3x^2}{1-\cos{x}}\Big)(1-\cos{x})}=e^{-\lim\limits_{x\rightarrow0} 3x^2}=e^0=1
\end{array}
$$
\\
\text{\bf(е):}
$$
\begin{array}{l}
\lim\limits_{x\rightarrow\pi} \dfrac{x^2-{\pi}^2}{\sin{x}}=\left[\dfrac{0}{0}\right]=\left|y=x-\pi,y\rightarrow0\right|=\lim\limits_{y\rightarrow0} \dfrac{\left(y+\pi\right)^2-{\pi}^2}{\sin({y}+\pi)}=\lim\limits_{y\rightarrow0} \dfrac {y(y+2\pi)} {-\sin{y}}=\left|y\rightarrow0,-\sin{y}\sim-{y}\right|=  \\=\lim\limits_{y\rightarrow0} -({y+2\pi})=-2\pi
\end{array}
$$
% ---------------------------- Problem 3----------------------------------
\subsubsection*{\center Задача № 3.}
{\bf Условие.~}\\
\text{\bf(а):} Показать, что данные функции
$f(x)$ и $g(x)$ являются бесконечно малыми или бесконечно большими
при указанном стремлении аргумента. \\
\text{\bf(б):} Для каждой функции $f(x)$ и $g(x)$ записать главную часть
(эквивалентную ей функцию)  вида $C(x-x_0)^{\alpha}$ при $x\rightarrow x_0$ или $Cx^{\alpha}$
при $x\rightarrow\infty$, указать их порядки малости (роста). \\
\text{\bf(в):} Сравнить функции $f(x)$ и $g(x)$ при указанном стремлении.
\begin{center}
	\begin{tabular}{|c|c|c|}
		\hline
		№ варианта & функции $f(x)$ и $g(x)$ & стремление \\[6pt]
		\hline
		19 & $f(x) = x^2+x-2,~g(x)=\dfrac{\ln(x+3)}{\arcsin{\sqrt{x+2}}}$ & $x\rightarrow-2+$ \\
		\hline
		\end{tabular}
		\bigskip
		\\
		{\bf Решение.~}\\
		\end{center}
		\medskip
		\text{\bf(а):}~Покажем, что $f(x)$ и $g(x)$ бесконечно малые функции,
$$
 \begin{array}{l} 
\lim\limits_{x\rightarrow-2+} f(x)=\lim\limits_{x\rightarrow -2+} x^2+x-2 =\lim\limits_{x\rightarrow -2+} (-2)^2-2-2=0 , \\
 
\lim\limits_{x\rightarrow-2+} g(x)= \lim\limits_{x\rightarrow -2+}  \dfrac{\ln(x+3)}{\arcsin{\sqrt{x+2}}}=\left|y=x+2,y\rightarrow0\right|=\lim\limits_{y\rightarrow 0} \dfrac{\ln(y+1)}{\arcsin\sqrt{y}}=\\
=\left|y\rightarrow0,\ln(y+1)\sim{y},\arcsin\sqrt{y}\sim\sqrt{y}\right|=\lim\limits_{y\rightarrow 0} \dfrac{y}{\sqrt{y}}=\lim\limits_{y\rightarrow 0} \sqrt{y}=0.
\end{array}
$$
\text{\bf(б):}~Так как $f(x)$ и $g(x)$ бесконечно малые функции, то эквивалентными им будут функции вида 
$C(x-x_0)^{\alpha}$ при $x\rightarrow x_0$. Найдём эквивалентную для $f(x)$ из условия
$$
\lim\limits_{x\rightarrow x_0}\dfrac{f(x)}{(x-x_0)^{\alpha}} = C,
$$
где $C$ --- некоторая константа. Рассмотрим предел
$$
\lim\limits_{x\rightarrow -2+} \dfrac{x^2+x-2}{(x+2)^{\alpha}}=\lim\limits_{x\rightarrow -2+} \dfrac{(x+2)(x-1)}{(x+2)^{\alpha}}=
\lim\limits_{x\rightarrow -2+} \dfrac{x-1}{(x+2)^{\alpha-1}}=\left[\alpha=1\right]=-3
$$
При $\alpha=1$ предел равен $-3$, отсюда $C=-3$ и 
$$
f(x)\sim -3x-6~\text{при}~x\rightarrow -2+.
$$
Аналогично, рассмотрим предел
$$
\lim\limits_{x\rightarrow x_0}\dfrac{g(x)}{(x-x_0)^{\alpha}} =\lim\limits_{x\rightarrow -2+}\dfrac{\ln(x+3)}{\arcsin\sqrt{x+2}(x+2)^{\alpha}}=\left|y=x+2,y\rightarrow0\right|=\lim\limits_{y\rightarrow 0} \dfrac{\ln(y+1)}{\arcsin\sqrt{y} (y^{\alpha})}=
$$
\newpage
$$
= \left|y\rightarrow0,\ln(1+y)\sim y,\arcsin\sqrt{y}\sim \sqrt{y}\right|=\lim\limits_{y\rightarrow 0} \dfrac{y}{\sqrt{y} (y^{\alpha})}= 
\lim\limits_{y\rightarrow 0} \dfrac{\sqrt{y}}{y^{\alpha}}=\left[\alpha=\dfrac{1}{2}\right]=1
$$
При $\alpha=\dfrac{1}{2}$ предел равен $1$, отсюда $C=1$ и
$$
g(x)\sim \sqrt{x+2} ~\text{при}~x\rightarrow -2+.
$$
\text{\bf(в):}~Для сравнения функций $f(x)$ и $g(x)$ рассмотрим предел их отношения при указанном стремлении
$$
\lim\limits_{x\rightarrow -2+}\dfrac{f(x)}{g(x)}.
$$
Применим эквивалентности, определенные в пункте (б), получим
$$
\lim\limits_{x\rightarrow -2+}\dfrac{f(x)}{g(x)} = \lim\limits_{x\rightarrow -2+}\dfrac{-3x-6}{\sqrt{x+2}}=\lim\limits_{x\rightarrow -2+} \dfrac{-3(x+2)}{\sqrt{x+2}}=\lim\limits_{x\rightarrow -2+} -3\sqrt{x+2}=0
$$
Отсюда, $f(x)$ есть бесконечно малая функция более высокого порядка роста, чем $g(x)$.
%=================================================================================================================================
%\subsection{Приложения дифференциального исчисления.}
%\input{src/part3.tex}
\newpage
\addcontentsline{toc}{section}{Список литературы}
\begin{thebibliography}{99}
\bibitem{book01} Львовский С.М. Набор и вёрстка в системе \LaTeX, 2003 c.
\bibitem{book02} Е.М. Балдин Компьютерная типография \LaTeX.
\bibitem{book03} М.С. Чебарыков Основы работы в системе \LaTeX, 2014.
\end{thebibliography}
\end{document}
